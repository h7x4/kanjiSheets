\usepackage[most]{tcolorbox}
\usepackage{xcolor}
\usepackage{anyfontsize}
\usepackage{tikz}

\definecolor{myGreen}{RGB}{72, 194, 78}

% ---------------------------------------------------------------------------- %
%                                 JLPT Section                                 %
% ---------------------------------------------------------------------------- %

\newcommand{\jlptSection}[1]{
  \section*{\uppercase{#1}}

  \resizebox{\textwidth}{!}{
    \input{./data/tables/#1.tex}
  }

  \localtableofcontents

  \break
}

% ---------------------------------------------------------------------------- %
%                                  Header line                                 %
% ---------------------------------------------------------------------------- %

\newcommand{\taughtIn}[1]{
  Taught in: #1 \newline
}

\newcommand{\jlptLevel}[1]{
  JLPT Level: #1 \newline
}

\newcommand{\strokeCount}[1]{
  Stroke count: #1 \newline
}

\newcommand{\kanji}[1]{
  \begin{center}
      \resizebox{\textwidth}{!}{
        \begin{tikzpicture}
          \fill[rounded corners=2pt, fill=myGreen, draw=black] (0,0) rectangle (1,1);
          \draw (0.5,0.23) node[white, anchor=base, scale=2]{#1};
        \end{tikzpicture}
    }
    \phantomsection
    \addcontentsline{toc}{subsection}{#1}
  \end{center}
}

\newcommand{\radicalHeader}[1]{
  \begin{center}
    \resizebox{.4\textwidth}{!}{
      \begin{tikzpicture}
        \draw[thick] (0,0) rectangle (1,1);
        \draw (0.5,0.23) node[anchor=base, scale=2]{#1};
      \end{tikzpicture}
    }
  \end{center}
}

\newcommand{\kanjiPageHeader}[5]{
  \begin{minipage}{0.3\textwidth}
    \begin{flushleft}
      \taughtIn{#2}
      \jlptLevel{#3}
      \strokeCount{#4}
    \end{flushleft}
  \end{minipage} \hfill
  \begin{minipage}{0.3\textwidth}
    \kanji{#1}
  \end{minipage} \hfill
  \begin{minipage}{0.3\textwidth}
    \begin{flushright}
      \radicalHeader{#5}
    \end{flushright}
  \end{minipage}
  \vspace{1cm}
}

% ---------------------------------------------------------------------------- %
%                                    Meaning                                   %
% ---------------------------------------------------------------------------- %

\newtcolorbox{meaningBox}{ %TODO: Make the box autoshrink horizontally while still making newlines when it's full.
  enhanced,
  colback=black!20,
  left=15pt,
  right=15pt,
  top=15pt,
  bottom=15pt,
  width=\textwidth,
  hbox,
  capture=minipage
}

\newcommand{\kanjiMeaning}[1]{
  \begin{center}
    \begin{meaningBox}
      \Large #1
    \end{meaningBox}
  \end{center}
}

% ---------------------------------------------------------------------------- %
%                              Kunyomi and Onyomi                              %
% ---------------------------------------------------------------------------- %

\newtcolorbox{kunyomiBox}{
  enhanced,
  attach boxed title to top left={yshift=-12pt,xshift=10pt},
  colback=blue!20,
  colframe=blue!60,
  left=15pt,
  right=15pt,
  top=15pt,
  bottom=5pt,
  title={\fontsize{15}{20}\textcolor{white}{\textbf{訓読み}}},
  colbacktitle=blue!60,
  width=\textwidth
}

\newcommand{\kunyomi}[1]{
  \begin{kunyomiBox}
    \fontsize{15}{20}
    #1 
  \end{kunyomiBox}
}

\newtcolorbox{onyomiBox}{
  enhanced,
  attach boxed title to top left={yshift=-12pt,xshift=10pt},
  colback=red!20,
  colframe=red!60,
  left=15pt,
  right=15pt,
  top=15pt,
  bottom=5pt,
  title={\fontsize{15}{20}\textcolor{white}{\textbf{音読み}}},
  colbacktitle=red!60,
  width=\textwidth,
}

\newcommand{\onyomi}[1]{
  \begin{onyomiBox}
    \fontsize{15}{20}
    \textbf{\textcolor{myGreen!80!black}{#1}}
  \end{onyomiBox}
  \vspace{0.5cm}
}

% ---------------------------------------------------------------------------- %
%                               Kanji Drawing Box                              %
% ---------------------------------------------------------------------------- %

\newCJKfontfamily\drawingKanji[
  Path = {./font/},
  Extension = .ttf,
]{Choumei}
\newCJKfontfamily\drawingFirstKanji[
  Path = {./font/},
  Extension = .ttf,
]{KanjiStrokeOrders}

\newcommand{\kanjiRow}[1]{
  \resizebox{\textwidth}{!}{
    \begin{tikzpicture}[font=\drawingKanji, anchor=base]
      \clip (-0.1,-0.1) rectangle (14.1,6.1);

      % Main Frame
      \draw[line width=1.6pt, rounded corners=4pt] (0, 0) rectangle (14, 6);

      % Horizontal lines
      \draw (0,2) -- (14,2);
      \draw (0,4) -- (14,4);
  
      % Vertical lines
      \draw (2,0) -- (2,6);
      \draw (4,0) -- (4,6);
      \draw (6,0) -- (6,6);
      \draw (8,0) -- (8,6);
      \draw (10,0) -- (10,6);
      \draw (12,0) -- (12,6);

      % Grid Top
      \draw[dashed, opacity=.5] (0,5) -- (14,5);
      \draw[dashed, opacity=.5] (1,4) -- (1,6);
      \draw[dashed, opacity=.5] (3,4) -- (3,6);
      \draw[dashed, opacity=.5] (5,4) -- (5,6);
      \draw[dashed, opacity=.5] (7,4) -- (7,6);
      \draw[dashed, opacity=.5] (9,4) -- (9,6);
      \draw[dashed, opacity=.5] (11,4) -- (11,6);
      \draw[dashed, opacity=.5] (13,4) -- (13,6);

      % Grid Bottom
      \draw(0,1) -- (14,1);
      \draw (1,0) -- (1,2);
      \draw (3,0) -- (3,2);
      \draw (5,0) -- (5,2);
      \draw (7,0) -- (7,2);
      \draw (9,0) -- (9,2);
      \draw (11,0) -- (11,2);
      \draw (13,0) -- (13,2);

      % Big Characters 
      \draw (0.95,4.42) node[scale=5, font=\drawingFirstKanji]{#1};
      \draw (3,4.3) node[scale=5, opacity=.2, inner sep=0pt]{#1};
      \draw (5,4.3) node[scale=5, opacity=.15]{#1};
      \draw (7,4.3) node[scale=5, opacity=.12]{#1};
      \draw (9,4.3) node[scale=5, opacity=.08]{#1};
      \draw (11,4.3) node[scale=5, opacity=.04]{#1};

      % Small characters
      \draw (0.5,1.15) node[scale=2]{#1};
      \draw (0.5,0.15) node[scale=2, opacity=.3]{#1};
      \draw (1.5,1.15) node[scale=2, opacity=.2]{#1};
      \draw (1.5,0.15) node[scale=2, opacity=.15]{#1};
      \draw (2.5,1.15) node[scale=2, opacity=.1]{#1};
      \draw (2.5,0.15) node[scale=2, opacity=.05]{#1};
  
    \end{tikzpicture}
  }
}