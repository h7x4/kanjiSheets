% !TEX program = xelatex

\documentclass{article}
\usepackage{hyperref}
\hypersetup{
    colorlinks=true,
    linkcolor=blue,
    filecolor=blue,      
    urlcolor=blue,
}

\usepackage{geometry}
\geometry{
  a4paper,
  left=20mm,
  right=20mm,
  top=20mm,
}

\setlength{\parskip}{0.5em}
\setlength{\parindent}{0pt}

\usepackage{longtable} %N1 table is too long.

%Japanese typesetting and fonts
\usepackage[japanese]{babel}
\usepackage{xeCJK}
\usepackage{fontspec}
\setCJKmainfont{Noto Sans Mono CJK JP}
\setmainfont{Open Sans}

\usepackage[table]{xcolor}
\usepackage[many]{tcolorbox}
\usepackage{anyfontsize}
\usepackage{tikz}

\definecolor{kanjiColor}{RGB}{72, 194, 78}
\colorlet{kunyomiColor}{blue}
\colorlet{onyomiColor}{red}
\colorlet{meaningColor}{black}

% ---------------------------------------------------------------------------- %
%                             Chapter Introduction                             %
% ---------------------------------------------------------------------------- %

\newcommand{\tocPiece}[2]{
  % For some reason, I'm not able to use \uppercase in order to use
  % a single argument. In the end, I found that just using two arguments
  % was going to be easier than trying to override the way the LaTeX
  % kernel handles this command.
  \section{#1}

  \begin{center}
    \fontsize{16}{16}
    \rowcolors{1}{}{kanjiColor!20!white}
    \input{./data/tables/#2.tex}
  \end{center}
}

\setcounter{secnumdepth}{0}

\newcommand{\chapterIntroduction}[2]{
  \tocPiece{#1}{#2}
  \break
}

% ---------------------------------------------------------------------------- %
%                                  Kanji Table                                 %
% ---------------------------------------------------------------------------- %

\newenvironment{chapterTabular}[1]
  {
    \begin{longtable}{#1}
  }
  {
    \end{longtable}
  }

% ---------------------------------------------------------------------------- %
%                                  Header line                                 %
% ---------------------------------------------------------------------------- %

\newcommand{\taughtIn}[1]{
  #1 \newline
}

\newcommand{\jlptLevel}[1]{
  JLPT Level: #1 \newline
}

\newcommand{\strokeCount}[1]{
  Stroke count: #1 \newline
}

\newcommand{\kanji}[1]{
  \begin{center}
      \resizebox{\textwidth}{!}{
        \begin{tikzpicture}
          \fill[rounded corners=2pt, fill=kanjiColor, draw=black] (0,0) rectangle (1,1);
          \draw (0.5,0.23) node[white, anchor=base, scale=2]{#1};
        \end{tikzpicture}
    }
    \phantomsection
    \addcontentsline{toc}{subsection}{#1}
  \end{center}
}

\newcommand{\radicalHeader}[1]{
  \begin{center}
    \resizebox{.4\textwidth}{!}{
      \begin{tikzpicture}
        \draw[thick] (0,0) rectangle (1,1);
        \draw (0.5,0.23) node[anchor=base, scale=2]{#1};
      \end{tikzpicture}
    }
  \end{center}
}

\newcommand{\kanjiPageHeader}[5]{
  \begin{minipage}{0.3\textwidth}
    \begin{flushleft}
      \taughtIn{#2}
      \jlptLevel{#3}
      \strokeCount{#4}
    \end{flushleft}
  \end{minipage} \hfill
  \begin{minipage}{0.3\textwidth}
    \kanji{#1}
  \end{minipage} \hfill
  \begin{minipage}{0.3\textwidth}
    \begin{flushright}
      \radicalHeader{#5}
    \end{flushright}
  \end{minipage}
  \vspace{1cm}
}

% ---------------------------------------------------------------------------- %
%                                    Meaning                                   %
% ---------------------------------------------------------------------------- %

\newtcolorbox{meaningBox}{
  tcbox width=auto limited,
  capture=hbox,
  enhanced,
  colback=meaningColor!20,
  left=15pt,
  right=15pt,
  top=15pt,
  bottom=15pt,
  arc=0pt,
  outer arc=0pt,
  attach boxed title to top center={yshift=-12pt, yshifttext=-6pt},
  title={\fontsize{15}{20}\textcolor{white}{\textbf{意味}}},
  colbacktitle=meaningColor!60,
  boxed title style={arc=0pt, outer arc=0pt}
}

\newcommand{\kanjiMeaning}[1]{
  \begin{center}
    \begin{meaningBox}
      \Large #1
    \end{meaningBox}
  \end{center}
}

% ---------------------------------------------------------------------------- %
%                              Kunyomi and Onyomi                              %
% ---------------------------------------------------------------------------- %

\newcommand{\emphasize}[1]{\textbf{\textcolor{kanjiColor!80!black}{#1}}}

\newtcolorbox{kunyomiBox}{
  enhanced,
  attach boxed title to top left={yshift=-12pt,xshift=10pt},
  colback=kunyomiColor!20,
  colframe=kunyomiColor!60,
  left=15pt,
  right=15pt,
  top=15pt,
  bottom=5pt,
  title={\fontsize{15}{20}\textcolor{white}{\textbf{訓読み}}},
  colbacktitle=kunyomiColor!60,
  width=\textwidth
}

\newcommand{\kunyomi}[1]{
  \begin{kunyomiBox}
    \fontsize{15}{20}
    #1 
  \end{kunyomiBox}
}

\newtcolorbox{onyomiBox}{
  enhanced,
  attach boxed title to top left={yshift=-12pt,xshift=10pt},
  colback=onyomiColor!20,
  colframe=onyomiColor!60,
  left=15pt,
  right=15pt,
  top=15pt,
  bottom=5pt,
  title={\fontsize{15}{20}\textcolor{white}{\textbf{音読み}}},
  colbacktitle=onyomiColor!60,
  width=\textwidth,
}

\newcommand{\onyomi}[1]{
  \begin{onyomiBox}
    \fontsize{15}{20}
    \textbf{\textcolor{kanjiColor!80!black}{#1}}
  \end{onyomiBox}
  \vspace{0.5cm}
}

% ---------------------------------------------------------------------------- %
%                               Kanji Drawing Box                              %
% ---------------------------------------------------------------------------- %

\newCJKfontfamily\drawingKanji[
  Path = {./font/},
  Extension = .ttf,
]{Choumei}
\newCJKfontfamily\drawingFirstKanji[
  Path = {./font/},
  Extension = .ttf,
]{KanjiStrokeOrders}

\newcommand{\kanjiRow}[1]{
  \resizebox{\textwidth}{!}{
    \begin{tikzpicture}[font=\drawingKanji, anchor=base]
      \clip (-0.1,-0.1) rectangle (14.1,6.1);

      % Main Frame
      \draw[line width=1.6pt] (0, 0) rectangle (14, 6);

      % Horizontal lines
      \draw (0,2) -- (14,2);
      \draw (0,4) -- (14,4);
  
      % Vertical lines
      \draw (2,0) -- (2,6);
      \draw (4,0) -- (4,6);
      \draw (6,0) -- (6,6);
      \draw (8,0) -- (8,6);
      \draw (10,0) -- (10,6);
      \draw (12,0) -- (12,6);

      % Grid Top
      \draw[dashed, opacity=.5] (0,5) -- (14,5);
      \draw[dashed, opacity=.5] (1,4) -- (1,6);
      \draw[dashed, opacity=.5] (3,4) -- (3,6);
      \draw[dashed, opacity=.5] (5,4) -- (5,6);
      \draw[dashed, opacity=.5] (7,4) -- (7,6);
      \draw[dashed, opacity=.5] (9,4) -- (9,6);
      \draw[dashed, opacity=.5] (11,4) -- (11,6);
      \draw[dashed, opacity=.5] (13,4) -- (13,6);

      % Grid Bottom
      \draw(0,1) -- (14,1);
      \draw (1,0) -- (1,2);
      \draw (3,0) -- (3,2);
      \draw (5,0) -- (5,2);
      \draw (7,0) -- (7,2);
      \draw (9,0) -- (9,2);
      \draw (11,0) -- (11,2);
      \draw (13,0) -- (13,2);

      % Big Characters 
      \draw (0.95,4.42) node[scale=5, font=\drawingFirstKanji]{#1};
      \draw (3,4.3) node[scale=5, opacity=.2, inner sep=0pt]{#1};
      \draw (5,4.3) node[scale=5, opacity=.15]{#1};
      \draw (7,4.3) node[scale=5, opacity=.12]{#1};
      \draw (9,4.3) node[scale=5, opacity=.08]{#1};
      \draw (11,4.3) node[scale=5, opacity=.04]{#1};

      % Small characters
      \draw (0.5,1.15) node[scale=2]{#1};
      \draw (0.5,0.15) node[scale=2, opacity=.3]{#1};
      \draw (1.5,1.15) node[scale=2, opacity=.2]{#1};
      \draw (1.5,0.15) node[scale=2, opacity=.15]{#1};
      \draw (2.5,1.15) node[scale=2, opacity=.1]{#1};
      \draw (2.5,0.15) node[scale=2, opacity=.05]{#1};
  
    \end{tikzpicture}
  }
}

\usepackage{etoc} %For local tocs containing level based kanji list.

\begin{document}


\newgeometry{left=-1cm, right=-1cm, bottom=-1cm, top=-1cm}

\resizebox{\textwidth}{!}{
  \begin{tikzpicture}
    \clip (-12,-14) rectangle (12,19.1);
    \draw (0,0) node {\setlength{\tabcolsep}{1pt}
\resizebox{2\textwidth}{!}{
\begin{tabular}{ c c c c c c c c c c c c c c c c c c c c c c c c c c c c c c c c c c c c c }
朕 & 織 & 垣 & 素 & 欧 & 堪 & 衛 & 疫 & 週 & 故 & 測 & 源 & 隔 & 錯 & 儀 & 斤 & 凡 & 円 & 褒 & 愁 & 急 & 収 & 漢 & 閉 & 憶 & 岐 & 肝 & 離 & 亭 & 軒 & 誌 & 便 & 園 & 械 & 魂 & 所 & 響 \\
奏 & 換 & 漆 & 閣 & 画 & 月 & 普 & 抵 & 溝 & 排 & 入 & 机 & 肯 & 駐 & 満 & 緯 & 准 & 整 & 償 & 迎 & 鏡 & 代 & 金 & 氷 & 法 & 厚 & 具 & 心 & 判 & 需 & 企 & 浴 & 済 & 旅 & 紹 & 号 & 小 \\
淡 & 門 & 資 & 捕 & 導 & 将 & 鬼 & 三 & 官 & 掃 & 蓄 & 旗 & 空 & 情 & 題 & 我 & 逐 & 選 & 洪 & 符 & 寛 & 漬 & 宵 & 反 & 毒 & 妄 & 盟 & 雇 & 英 & 避 & 減 & 刑 & 壮 & 卑 & 衣 & 篤 & 病 \\
猿 & 飽 & 阻 & 扱 & 泉 & 奴 & 天 & 双 & 退 & 僕 & 粗 & 陽 & 撲 & 怠 & 区 & 尿 & 丈 & 侵 & 嫌 & 乳 & 寂 & 脱 & 図 & 造 & 駅 & 場 & 観 & 脚 & 閲 & 失 & 握 & 戸 & 悟 & 陪 & 帝 & 睡 & 精 \\
壊 & 迫 & 逮 & 流 & 少 & 挙 & 后 & 夫 & 亜 & 据 & 焦 & 員 & 禍 & 独 & 玉 & 渦 & 香 & 救 & 私 & 丙 & 宅 & 庸 & 介 & 幼 & 犠 & 恵 & 糖 & 肢 & 誘 & 歓 & 渡 & 券 & 統 & 追 & 始 & 野 & 扶 \\
遭 & 誕 & 患 & 奉 & 四 & 和 & 析 & 杯 & 丸 & 転 & 原 & 岬 & 焼 & 片 & 功 & 秒 & 舎 & 板 & 滝 & 等 & 組 & 倒 & 灰 & 利 & 尊 & 振 & 邸 & 羅 & 舟 & 税 & 廉 & 論 & 征 & 麗 & 降 & 江 & 槽 \\
渉 & 訪 & 建 & 謙 & 去 & 硫 & 技 & 著 & 説 & 束 & 芽 & 燃 & 初 & 漁 & 芝 & 黙 & 農 & 歌 & 墳 & 究 & 花 & 施 & 亡 & 葬 & 殿 & 汗 & 湾 & 至 & 援 & 巡 & 斜 & 義 & 央 & 水 & 移 & 窓 & 嚇 \\
算 & 全 & 悼 & 鋳 & 圏 & 要 & 囚 & 演 & 湯 & 科 & 序 & 秀 & 殴 & 紫 & 料 & 箱 & 連 & 虚 & 廊 & 曲 & 混 & 迷 & 墨 & 妻 & 遠 & 玄 & 憂 & 住 & 橋 & 脹 & 耳 & 更 & 能 & 七 & 製 & 半 & 妙 \\
冷 & 県 & 恐 & 傑 & 預 & 井 & 踏 & 客 & 鮮 & 沖 & 冊 & 艇 & 荘 & 栄 & 略 & 晴 & 向 & 赴 & 紀 & 壱 & 討 & 互 & 夜 & 励 & 仏 & 台 & 汽 & 誓 & 旨 & 任 & 端 & 居 & 姓 & 制 & 弐 & 扇 & 基 \\
鶏 & 襲 & 貫 & 衆 & 級 & 由 & 愉 & 窒 & 福 & 副 & 瞬 & 塀 & 拓 & 耕 & 粛 & 貞 & 備 & 背 & 紺 & 修 & 映 & 致 & 裸 & 既 & 慎 & 負 & 進 & 環 & 緒 & 終 & 役 & 震 & 喫 & 悠 & 幸 & 惨 & 融 \\
凝 & 植 & 破 & 嫁 & 帳 & 傘 & 価 & 径 & 劣 & 起 & 光 & 紅 & 枚 & 博 & 炊 & 銑 & 殻 & 吐 & 緊 & 芸 & 耐 & 努 & 秋 & 酷 & 紛 & 騎 & 轄 & 訳 & 肖 & 都 & 徒 & 件 & 痛 & 刀 & 仲 & 酵 & 酒 \\
人 & 御 & 着 & 症 & 獣 & 語 & 裏 & 弦 & 険 & 公 & 堅 & 含 & 乙 & 朽 & 粉 & 壁 & 凸 & 諮 & 鈴 & 宮 & 社 & 貢 & 南 & 執 & 騰 & 衡 & 軍 & 丹 & 洗 & 僚 & 倍 & 歴 & 使 & 閑 & 格 & 滋 & 島 \\
拡 & 封 & 式 & 架 & 術 & 悲 & 布 & 彫 & 媒 & 譜 & 早 & 聖 & 匿 & 酌 & 且 & 枢 & 学 & 話 & 藩 & 町 & 登 & 脂 & 憎 & 皇 & 嫡 & 凍 & 繊 & 郡 & 妊 & 続 & 抹 & 識 & 爵 & 訓 & 趣 & 色 & 処 \\
疲 & 康 & 族 & 及 & 夕 & 宙 & 棺 & 戯 & 訴 & 聞 & 診 & 諸 & 俵 & 郵 & 放 & 仰 & 争 & 循 & 坑 & 民 & 中 & 筒 & 隆 & 名 & 改 & 剰 & 慌 & 賞 & 勢 & 祖 & 越 & 煙 & 率 & 置 & 得 & 者 & 物 \\
戒 & 慨 & 屋 & 君 & 逃 & 走 & 了 & 泊 & 塊 & 枝 & 引 & 厄 & 殊 & 擦 & 磨 & 止 & 印 & 左 & 嘱 & 驚 & 虐 & 表 & 角 & 薄 & 嗣 & 貨 & 養 & 実 & 刈 & 魅 & 家 & 笛 & 推 & 陣 & 弔 & 洋 & 毎 \\
深 & 傷 & 医 & 特 & 道 & 幻 & 操 & 訟 & 奮 & 軟 & 縄 & 宜 & 配 & 仮 & 氏 & 臣 & 輩 & 固 & 永 & 田 & 遣 & 繕 & 杉 & 予 & 領 & 太 & 群 & 劾 & 但 & 髪 & 述 & 友 & 租 & 留 & 議 & 勘 & 昨 \\
癒 & 糸 & 馬 & 抄 & 詞 & 偽 & 邪 & 酬 & 二 & 羊 & 国 & 張 & 貯 & 唆 & 際 & 易 & 猛 & 曜 & 遮 & 剛 & 巣 & 柔 & 揚 & 鉛 & 師 & 余 & 啓 & 過 & 複 & 悔 & 胸 & 塩 & 昆 & 試 & 司 & 願 & 齢 \\
祝 & 姿 & 廷 & 沼 & 辞 & 微 & 属 & 捜 & 夏 & 力 & 近 & 昼 & 直 & 宝 & 貿 & 宴 & 刺 & 路 & 挑 & 細 & 唯 & 賓 & 翼 & 勅 & 棒 & 該 & 強 & 霊 & 況 & 寺 & 総 & 錘 & 披 & 憲 & 呈 & 祉 & 西 \\
妹 & 裁 & 殺 & 霜 & 肺 & 根 & 桟 & 畔 & 桜 & 共 & 雅 & 措 & 卸 & 高 & 元 & 絹 & 遍 & 徳 & 魚 & 賦 & 油 & 摩 & 綱 & 解 & 徹 & 卒 & 尼 & 裕 & 召 & 勤 & 拾 & 非 & 性 & 伺 & 材 & 盛 & 通 \\
木 & 坊 & 今 & 繭 & 神 & 孫 & 汚 & 則 & 筆 & 剤 & 影 & 拝 & 陶 & 劇 & 雪 & 若 & 査 & 街 & 鋼 & 綿 & 泡 & 郊 & 比 & 暴 & 像 & 飢 & 令 & 思 & 稲 & 湖 & 吟 & 膨 & 短 & 渇 & 完 & 困 & 璽 \\
注 & 簿 & 吹 & 恥 & 載 & 示 & 冬 & 輝 & 胆 & 係 & 繰 & 再 & 娠 & 峡 & 疾 & 朴 & 親 & 翁 & 鋭 & 詳 & 穂 & 末 & 痴 & 塗 & 紡 & 妥 & 村 & 数 & 歯 & 装 & 拒 & 政 & 嬢 & 継 & 澄 & 呉 & 里 \\
責 & 偉 & 豪 & 対 & 矢 & 升 & 長 & 害 & 賃 & 運 & 抑 & 験 & 活 & 遺 & 一 & 浮 & 戻 & 硬 & 苗 & 桃 & 習 & 営 & 較 & 富 & 秩 & 堂 & 六 & 獲 & 景 & 監 & 乱 & 松 & 詔 & 鉢 & 陛 & 生 & 干 \\
教 & 船 & 粘 & 壌 & 付 & 克 & 績 & 乗 & 星 & 把 & 沈 & 波 & 治 & 遅 & 後 & 吸 & 探 & 受 & 孤 & 渓 & 効 & 静 & 堀 & 班 & 潮 & 圧 & 最 & 辺 & 守 & 庭 & 猟 & 協 & 愛 & 暁 & 勺 & 港 & 遂 \\
濯 & 候 & 範 & 棚 & 鳥 & 希 & 塚 & 丘 & 感 & 不 & 鉄 & 充 & 距 & 傍 & 寧 & 携 & 手 & 様 & 偵 & 容 & 奔 & 泰 & 箇 & 坂 & 釈 & 津 & 誤 & 穴 & 事 & 経 & 斉 & 砲 & 惰 & 錬 & 青 & 拍 & 免 \\
射 & 鎖 & 支 & 暦 & 兵 & 顧 & 作 & 巧 & 託 & 占 & 帯 & 臨 & 域 & 晶 & 薦 & 払 & 派 & 集 & 未 & 麻 & 腕 & 詩 & 屯 & 赤 & 盤 & 鉱 & 碑 & 斗 & 計 & 膜 & 倹 & 滅 & 紋 & 畝 & 奥 & 骨 & 婆 \\
承 & 復 & 聴 & 信 & 撃 & 欠 & 鈍 & 府 & 団 & 豆 & 校 & 繁 & 局 & 外 & 遷 & 世 & 粧 & 本 & 雷 & 貧 & 瓶 & 疑 & 腸 & 順 & 畜 & 郎 & 娘 & 針 & 給 & 商 & 伝 & 棄 & 掘 & 敬 & 抗 & 佐 & 尉 \\
餓 & 褐 & 尺 & 没 & 期 & 捨 & 漠 & 城 & 知 & 漫 & 万 & 季 & 増 & 扉 & 祈 & 縫 & 俳 & 許 & 時 & 系 & 銀 & 覆 & 定 & 絶 & 頑 & 株 & 美 & 詠 & 誉 & 朱 & 随 & 索 & 弧 & 午 & 孔 & 低 & 禅 \\
硝 & 懸 & 飼 & 準 & 球 & 航 & 築 & 揺 & 窯 & 辱 & 輸 & 達 & 浄 & 米 & 婦 & 殉 & 跳 & 吏 & 助 & 院 & 粋 & 恋 & 幣 & 詰 & 剖 & 礁 & 懇 & 僧 & 掲 & 寮 & 雄 & 出 & 品 & 例 & 署 & 苦 & 款 \\
番 & 陳 & 甘 & 誇 & 顔 & 陵 & 相 & 浜 & 軌 & 値 & 貴 & 古 & 危 & 革 & 牧 & 禁 & 姻 & 斎 & 叙 & 何 & 駄 & 祥 & 恨 & 枠 & 漸 & 匹 & 喚 & 拘 & 有 & 抱 & 峰 & 幾 & 目 & 冗 & 東 & 岳 & 諭 \\
如 & 下 & 百 & 状 & 詐 & 蛮 & 現 & 為 & 棟 & 絡 & 岸 & 舶 & 髄 & 研 & 寸 & 兄 & 遊 & 維 & 臭 & 威 & 匁 & 海 & 娯 & 平 & 歳 & 墾 & 審 & 逝 & 父 & 墓 & 標 & 慶 & 前 & 声 & 糾 & 宰 & 胎 \\
王 & 岩 & 務 & 損 & 森 & 撤 & 醜 & 飯 & 宗 & 位 & 店 & 記 & 陸 & 書 & 芋 & 参 & 艦 & 体 & 点 & 枯 & 哲 & 墜 & 忌 & 寄 & 眠 & 見 & 先 & 恭 & 割 & 担 & 煩 & 包 & 八 & 消 & 怪 & 零 & 敏 \\
発 & 好 & 工 & 泥 & 案 & 輪 & 巻 & 言 & 単 & 酢 & 盾 & 姉 & 賛 & 庫 & 潜 & 網 & 旋 & 荒 & 評 & 雌 & 凶 & 虞 & 延 & 地 & 欲 & 畳 & 採 & 史 & 憩 & 告 & 投 & 応 & 返 & 間 & 偏 & 厳 & 賜 \\
黒 & 崇 & 宿 & 曹 & 違 & 厘 & 沸 & 軽 & 働 & 折 & 主 & 謄 & 殖 & 巨 & 突 & 興 & 縮 & 版 & 競 & 弊 & 課 & 可 & 塔 & 蛇 & 懐 & 謡 & 視 & 奇 & 摘 & 賢 & 搾 & 帽 & 約 & 毛 & 剣 & 在 & 侮 \\
吉 & 徴 & 烈 & 偶 & 刻 & 薫 & 侯 & 身 & 化 & 意 & 業 & 勇 & 銃 & 供 & 白 & 筋 & 寿 & 悪 & 横 & 票 & 伴 & 缶 & 拙 & 菊 & 懲 & 貝 & 酸 & 漂 & 行 & 犯 & 塁 & 爆 & 健 & 茂 & 敢 & 勝 & 恩 \\
倫 & 益 & 培 & 暇 & 同 & 報 & 括 & 悩 & 舗 & 昭 & 欄 & 階 & 写 & 専 & 荷 & 狩 & 各 & 男 & 祭 & 首 & 伯 & 楼 & 甲 & 臓 & 婿 & 市 & 問 & 忘 & 以 & 泣 & 揮 & 柄 & 催 & 隠 & 財 & 認 & 迅 \\
防 & 安 & 糧 & 頂 & 唇 & 肩 & 窃 & 菜 & 獄 & 命 & 息 & 仙 & 躍 & 兆 & 灯 & 依 & 彩 & 察 & 翌 & 士 & 切 & 育 & 慮 & 抜 & 線 & 秘 & 柱 & 暑 & 棋 & 省 & 童 & 誠 & 土 & 蔵 & 提 & 笑 & 招 \\
超 & 頭 & 慕 & 卵 & 周 & 魔 & 鳴 & 窮 & 暖 & 勧 & 寒 & 埋 & 鯨 & 替 & 幽 & 関 & 仁 & 傾 & 熟 & 境 & 霧 & 春 & 被 & 堕 & 残 & 蒸 & 林 & 脈 & 粒 & 照 & 席 & 挟 & 五 & 茶 & 燥 & 珍 & 泳 \\
持 & 成 & 適 & 坪 & 谷 & 段 & 涙 & 湿 & 常 & 跡 & 矯 & 賊 & 壇 & 賀 & 忠 & 廃 & 妃 & 子 & 母 & 衷 & 汁 & 川 & 積 & 皿 & 銭 & 皮 & 薬 & 落 & 千 & 肌 & 覇 & 就 & 弁 & 称 & 峠 & 唐 & 暗 \\
逓 & 舌 & 雑 & 真 & 与 & 脳 & 撮 & 嘆 & 日 & 韻 & 幅 & 垂 & 考 & 却 & 攻 & 麦 & 河 & 沢 & 飛 & 温 & 回 & 倣 & 慈 & 暮 & 弓 & 念 & 純 & 畑 & 溶 & 孝 & 叔 & 喪 & 伐 & 別 & 刃 & 読 & 型 \\
恒 & 釣 & 冠 & 戦 & 鑑 & 沿 & 督 & 請 & 般 & 年 & 騒 & 委 & 掛 & 的 & 届 & 駆 & 籍 & 穫 & 権 & 併 & 幹 & 異 & 崎 & 藻 & 字 & 従 & 欺 & 浪 & 愚 & 擁 & 億 & 途 & 碁 & 理 & 交 & 部 & 奨 \\
旬 & 抽 & 斥 & 舞 & 購 & 調 & 拠 & 癖 & 触 & 削 & 批 & 錠 & 車 & 遵 & 然 & 礼 & 楽 & 添 & 膚 & 房 & 京 & 鼻 & 慰 & 差 & 濫 & 看 & 形 & 咲 & 確 & 覧 & 帰 & 襟 & 用 & 宣 & 袋 & 憾 & 量 \\
律 & 激 & 会 & 腹 & 立 & 闘 & 緩 & 談 & 老 & 明 & 冒 & 概 & 潔 & 質 & 頻 & 献 & 暫 & 否 & 伏 & 彼 & 擬 & 死 & 渋 & 策 & 絞 & 敷 & 動 & 洞 & 謝 & 虜 & 眼 & 徐 & 重 & 彰 & 淑 & 郷 & 穏 \\
忙 & 石 & 謹 & 面 & 囲 & 果 & 大 & 銘 & 尾 & 豊 & 層 & 寡 & 犬 & 九 & 炭 & 歩 & 肉 & 崩 & 女 & 隣 & 指 & 腐 & 翻 & 座 & 朝 & 札 & 掌 & 飾 & 設 & 並 & 顕 & 縛 & 琴 & 勲 & 透 & 節 & 屈 \\
寝 & 履 & 赦 & 滑 & 狂 & 栽 & 虫 & 度 & 竹 & 無 & 類 & 必 & 核 & 菌 & 授 & 似 & 隷 & 志 & 敵 & 呼 & 弱 & 方 & 条 & 練 & 個 & 十 & 規 & 貸 & 賠 & 編 & 稼 & 肪 & 賄 & 打 & 産 & 芳 & 久 \\
縁 & 北 & 譲 & 唱 & 酪 & 護 & 善 & 俸 & 即 & 庁 & 昔 & 他 & 眺 & 結 & 内 & 速 & 党 & 猶 & 怒 & 珠 & 華 & 押 & 正 & 皆 & 炎 & 職 & 胃 & 音 & 鼓 & 享 & 種 & 滴 & 曇 & 炉 & 濁 & 践 & 浦 \\
展 & 疎 & 取 & 隊 & 児 & 因 & 休 & 縦 & 多 & 快 & 申 & 露 & 草 & 栓 & 諾 & 陰 & 妨 & 痢 & 紙 & 納 & 想 & 覚 & 贈 & 床 & 閥 & 羽 & 望 & 電 & 叫 & 敗 & 存 & 新 & 又 & 象 & 桑 & 盗 & 待 \\
某 & 答 & 陥 & 肥 & 館 & 額 & 隻 & 砂 & 慢 & 断 & 盲 & 典 & 決 & 仕 & 療 & 合 & 染 & 罰 & 刷 & 罪 & 鎮 & 猫 & 池 & 濃 & 菓 & 態 & 怖 & 雲 & 豚 & 凹 & 勉 & 靴 & 腰 & 雰 & 味 & 択 & 頒 \\
熱 & 絵 & 労 & 買 & 涯 & 風 & 遇 & 胴 & 蛍 & 訂 & 描 & 警 & 尋 & 自 & 創 & 密 & 売 & 文 & 浸 & 旧 & 迭 & 礎 & 累 & 磁 & 搭 & 災 & 拐 & 俗 & 穀 & 障 & 証 & 刊 & 甚 & 補 & 幕 & 分 & 浅 \\
奪 & 州 & 帆 & 弟 & 裂 & 送 & 摂 & 踊 & 弾 & 晩 & 停 & 優 & 俊 & 良 & 軸 & 宇 & 夢 & 搬 & 己 & 罷 & 火 & 控 & 儒 & 附 & 液 & 求 & 器 & 契 & 葉 & 兼 & 簡 & 項 & 逸 & 側 & 滞 & 脅 & 伸 \\
潟 & 挿 & 悦 & 邦 & 章 & 帥 & 哀 & 足 & 第 & 講 & 憤 & 債 & 是 & 費 & 梅 & 雨 & 募 & 塑 & 接 & 黄 & 服 & 除 & 喜 & 樹 & 込 & 句 & 飲 & 鐘 & 堤 & 朗 & 加 & 卓 & 難 & 丁 & 両 & 気 & 才 \\
清 & 右 & 締 & 頼 & 塾 & 当 & 均 & 乾 & 蚕 & 佳 & 銅 & 漏 & 口 & 界 & 惑 & 検 & 酔 & 血 & 醸 & 到 & 逆 & 潤 & 尚 & 盆 & 慣 & 砕 & 辛 & 蚊 & 来 & 婚 & 忍 & 牲 & 販 & 衰 & 次 & 武 & 上 \\
列 & 室 & 紳 & 狭 & 程 & 昇 & 煮 & 散 & 耗 & 衝 & 噴 & 痘 & 鍛 & 瀬 & 開 & 謀 & 喝 & 尽 & 謁 & 管 & 借 & 稚 & 隅 & 薪 & 変 & 還 & 保 & 侍 & 匠 & 郭 & 緑 & 乏 & 模 & 機 & 姫 & 竜 & 拷
\end{tabular}
}};
    \draw(0,8.15) node {
      \resizebox{0.435\textwidth}{!}{
        \begin{tikzpicture}
          \fill[white] (0,0) rectangle (1,1);
          \draw (0.5,0.7) node{漢字};
          \draw (0.5,0.3) node{練習};
        \end{tikzpicture}
      }          
    };
  \end{tikzpicture}
}




% \resizebox{\textwidth}{!}{
%   \begin{tikzpicture}
%     \draw (0,0.4) node{漢字};
%     \draw (0,0) node{練習};
%     \draw()
%   \end{tikzpicture}
% }

\restoregeometry

\break

\tableofcontents
\break

\section*{Preface}

%TODO: Add more detailed information

Thanks to Timothy Eyre for making the Choumei font and the edition containing stroke order numbers. You can find his work at \url{https://www.nihilist.org.uk/}

Kanji data is taken from \url{https://jisho.org/}

This document splits the kanjis into JLPT levels. Please note that there is no official list of kanji JLPT levels. This list is based on \href{http://www.tanos.co.uk/jlpt/skills/kanji/}{tanos.co.uk}.

\break

\jlptSection{n5}

\input{./data/testing/test.tex}

\jlptSection{n4}

% \input{./data/pages/n4.tex}

\jlptSection{n3}

% \input{./data/pages/n3.tex}

\jlptSection{n2}

% \input{./data/pages/n2.tex}

\jlptSection{n1}

% \input{./data/pages/n1.tex}

\end{document}